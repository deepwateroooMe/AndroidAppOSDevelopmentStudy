% Created 2021-12-20 Mon 22:21
\documentclass[9pt, b5paper]{article}
\usepackage[UTF8]{ctex}
\usepackage{xltxtra}
\usepackage{bera}
\usepackage[T1]{fontenc}
\usepackage[scaled]{beraserif}
\usepackage[scaled]{berasans}
\usepackage[scaled]{beramono}
\usepackage{graphicx}
\usepackage{xcolor}
\usepackage{multirow}
\usepackage{multicol}
\usepackage{float}
\usepackage{textcomp}
\usepackage{geometry}
\geometry{left=1.2cm,right=1.2cm,top=1.5cm,bottom=1.2cm}
\usepackage{algorithm}
\usepackage{algorithmic}
\usepackage{latexsym}
\usepackage{natbib}
\usepackage{minted}
\newminted{common-lisp}{fontsize=ootnotesize}
\usepackage[xetex,colorlinks=true,CJKbookmarks=true,linkcolor=blue,urlcolor=blue,menucolor=blue]{hyperref}
\author{deepwaterooo}
\date{\today}
\title{Retrofit}
\hypersetup{
  pdfkeywords={},
  pdfsubject={},
  pdfcreator={Emacs 27.1 (Org mode 8.2.7c)}}
\begin{document}

\maketitle
\tableofcontents


\section{一、概述}
\label{sec-1}
\begin{itemize}
\item 1、什么是retrofit
\begin{itemize}
\item retrofit是现在比较流行的网络请求框架,可以理解为okhttp的加强版,底层封装了Okhttp。准确来说,Retrofit是一个RESTful的http网络请求框架的封装。因为网络请求工作本质上是由okhttp来完成,而Retrofit负责网络请求接口的封装。
\item 本质过程:App应用程序通过Retrofit请求网络,实质上是使用Retrofit接口层封装请求参数、Header、Url等信息,之后由okhttp来完成后续的请求工作。在服务端返回数据后,okhttp将原始数据交给Retrofit,Retrofit根据用户需求解析。(源码在文章最后给出)
\end{itemize}
\item 2、Retrofit的优点
\item 超级解耦 ,接口定义、接口参数、接口回调不在耦合在一起
\item 可以配置不同的httpClient来实现网络请求,如okhttp、httpclient
\item 支持同步、异步、Rxjava
\item 可以配置不同反序列化工具类来解析不同的数据,如json、xml
\item 请求速度快,使用方便灵活简洁
\end{itemize}
\subsection{二、注解}
\label{sec-1-1}
\begin{itemize}
\item Retrofit使用大量注解来简化请求,Retrofit将okhttp请求抽象成java接口,使用注解来配置和描述网络请求参数。大概可以分为以下几类,我们先来看看各个注解的含义,再一一去实践解释。
\end{itemize}
\subsubsection{1、请求方法注解}
\label{sec-1-1-1}
\begin{center}
\begin{tabular}{ll}
\hline
请求方法注解 & 说明\\
\hline
@GET & get请求\\
@POST & post请求\\
@PUT & put请求\\
@DELETE & delete请求\\
@PATCH & patch请求,该请求是对put请求的补充,用于更新局部资源\\
@HEAD & head请求\\
@OPTIONS & options请求\\
@HTTP & 通过注解,可以替换以上所有的注解,它拥有三个属性:method、path、hasBody\\
\hline
\end{tabular}
\end{center}
\subsubsection{2、请求头注解}
\label{sec-1-1-2}
\begin{center}
\begin{tabular}{ll}
\hline
请求头注解 & 说明\\
\hline
@Headers & 用于添加固定请求头,可同时添加多个,通过该注解的请求头不会相互覆盖,而是共同存在\\
@Header & 作为方法的参数传入,用于添加不固定的header,它会更新已有请求头\\
\hline
\end{tabular}
\end{center}
\subsubsection{3、请求参数注解}
\label{sec-1-1-3}
\begin{center}
\begin{tabular}{ll}
\hline
请求参数注解 & 说明\\
\hline
@Body & 多用于Post请求发送非表达数据,根据转换方式将实例对象转化为对应字符串传递参数,\\
 & 比如使用Post发送Json数据,添加GsonConverterFactory则是将body转化为json字符串进行传递\\
@Filed & 多用于Post方式传递参数,需要结合@FromUrlEncoded使用,即以表单的形式传递参数\\
@FiledMap & 多用于Post请求中的表单字段,需要结合@FromUrlEncoded使用\\
@Part & 用于表单字段,Part和PartMap与@multipart注解结合使用,适合文件上传的情况\\
@PartMap & 用于表单字段,默认接受类型是Map<String,RequestBody>,可用于实现多文件上传\\
@Path & 用于Url中的占位符\\
@Query & 用于Get请求中的参数\\
@QueryMap & 与Query类似,用于不确定表单参数\\
@Url & 指定请求路径\\
\hline
\end{tabular}
\end{center}
4、请求和响应格式(标记)注解
\begin{center}
\begin{tabular}{}
\end{tabular}
\end{center}
标记类注解    说明
\begin{center}
\begin{tabular}{ll}
\hline
@FromUrlCoded & 表示请求发送编码表单数据,每个键值对需要使用@Filed注解\\
@Multipart & 表示请求发送form$_{\text{encoded数据}}$(使用于有文件上传的场景),\\
 & 每个键值对需要用@Part来注解键名,随后的对象需要提供值\\
@Streaming & 表示响应用字节流的形式返回,如果没有使用注解,默认会把数据全部载入到\\
 & 内存中,该注解在下载大文件时特别有用\\
\hline
\end{tabular}
\end{center}

\subsection{Retrofit2使用详解}
\label{sec-1-2}
\begin{itemize}
\item 添加依赖:
\end{itemize}
\begin{minted}[frame=lines,fontsize=\scriptsize,linenos=false]{groovy}
 compile 'com.squareup.retrofit2:retrofit:2.0.2'
 compile 'com.squareup.retrofit2:converter-gson:2.0.2'
\end{minted}
\begin{itemize}
\item 因为Retrofit2是依赖okHttp请求的,而且请查看它的META-INF->META-INF\maven\com.squareup.retrofit2\retrofit->pom.xml文件
\begin{itemize}
\item \uline{这个文件是我不曾关注、不曾注意到过的,需要熟悉一下}
\end{itemize}
\end{itemize}
\begin{minted}[frame=lines,fontsize=\scriptsize,linenos=false]{xml}
<dependencies>
    <dependency>
      <groupId>com.squareup.okhttp3</groupId>
      <artifactId>okhttp</artifactId>
    </dependency>
</dependencies>
\end{minted}
\begin{itemize}
\item 由此可见,它确实是依赖okHttp,okHttp有会依赖okio所以它会制动的把这两个包也导入进来。
\item 添加权限:
\begin{itemize}
\item 既然要请求网络,在我们android手机上是必须要有访问网络的权限的,下面把权限添加进来
\end{itemize}
\begin{minted}[frame=lines,fontsize=\scriptsize,linenos=false]{xml}
<uses-permission android:name="android.permission.INTERNET"/>
\end{minted}
\item 好了,下面开始介绍怎么使用Retrofit,既然它是使用注解的请求方式来完成请求URL的拼接,那么我们就按注解的不同来分别学习:
\item 首先,我们需要创建一个java接口,用于存放请求方法的:
\end{itemize}
\begin{minted}[frame=lines,fontsize=\scriptsize,linenos=false]{java}
public interface GitHubService {
}
\end{minted}
\begin{itemize}
\item 然后逐步在该方法中添加我们所需要的方法(按照请求方式):
\end{itemize}
\subsubsection{1 :Get : 是我们最常见的请求方法,它是用来获取数据请求的。}
\label{sec-1-2-1}
\begin{enumerate}
\item ①:直接通过URL获取网络内容:
\label{sec-1-2-1-1}
\begin{minted}[frame=lines,fontsize=\scriptsize,linenos=false]{java}
public interface GitHubService {
    @GET("users/octocat/repos")
    Call<List<Repo>> listRepos();
}
\end{minted}
\begin{itemize}
\item 在这里我们定义了一个listRepos()的方法,通过@GET注解标识为get请求,请求的URL为“users/octocat/repos”。
\item 然后看看Retrofit是怎么调用的,代码如下:
\end{itemize}
\begin{minted}[frame=lines,fontsize=\scriptsize,linenos=false]{java}
Retrofit retrofit = new Retrofit.Builder()
    .baseUrl("https://api.github.com/") // 通过动态代理获取到所定义的接口
    .build();
GitHubService service = retrofit.create(GitHubService.class);
Call<List<Repo>> repos = service.listRepos();
repos.enqueue(new Callback<List<Repo>>(){
    @Override
    public void onResponse(Call<List<Repo>> call, Response<List<Repo>> response){
    }
    @Override
    public void onFailure(Call<List<Repo>> call, Throwable t){
    }
});
\end{minted}
\begin{itemize}
\item 代码解释:首先获取Retrofit对象,然后通过动态代理获取到所定义的接口,通过调用接口里面的方法获取到Call类型返回值,最后进行网络请求操作(这里不详细说明Retrofit 实现原理,后面会对它进行源码解析),这里必须要说的是请求URL的拼接:在构建Retrofit对象时调用baseUrl所传入一个String类型的地址,这个地址在调用service.listRepos()时会把@GET(“users/octocat/repos”)的URL拼接在尾部。
\end{itemize}
ok,这样就完成了,我们这次的请求,但是我们不能每次请求都要创建一个方法呀?这时我们就会想起 \uline{动态的构建URL} 了
\item @Path: ②:动态获取URL地址:@Path
\label{sec-1-2-1-2}
\begin{itemize}
\item 我们再上面的基础上进行修改,如下:
\end{itemize}
\begin{minted}[frame=lines,fontsize=\scriptsize,linenos=false]{java}
public interface GitHubService {
  @GET("users/{user}/repos")
  Call<List<Repo>> listRepos(@Path("user") String user);
\end{minted}
\begin{itemize}
\item 这里在Get注解中包含\{user\},它所对应的是@Path注解中的“user”,它所标示的正是String user,而我们再使用Retrofit对象动态代理的获取到GitHubService,当调用listRepos时,我们就必须传入一个String类型的User,如:
\end{itemize}
\begin{minted}[frame=lines,fontsize=\scriptsize,linenos=false]{java}
Call<List<Repo>> repos = service.listRepos("octocat");
\end{minted}
\begin{itemize}
\item 如上代码,其他的代码都是不变的,而我们只需要使用@Path注解就完全的实现了动态的URL地址了,是不是很方便呢,这还不算什么,通常情况下,我们去获取一些网络信息,因为信息量太大,我们会分类去获取,也就是携带一些必要的元素进行过滤,那我们该怎么实现呢?其实也很简单,因为Retrofit已经为我们封装好了注解,请看下面(官网实例):
\end{itemize}
\item @Query: ③:动态指定条件获取信息:@Query
\label{sec-1-2-1-3}
\begin{minted}[frame=lines,fontsize=\scriptsize,linenos=false]{java}
@GET("group/{id}/users")
Call<List<User>> groupList(@Path("id") int groupId, @Query("sort") String sort);
\end{minted}
\begin{itemize}
\item 我们只需要使用@Query注解即可完成我们的需求,在@Query(“sort”)中,short就好比是URL请求地址中的键,而它说对应的String sort中的sort则是它的值。
\item 但是我们想,在网络请求中一般为了更精确的查找到我们所需要的数据,过滤更多不需要的无关的东西,我们往往需要携带多个请求参数,当然可以使用@Query注解,但是太麻烦,很长,容易遗漏和出错,那有没有更简单的方法呢,有,当然后,我们可以直接放到一个map键值对中:
\end{itemize}
\item @QueryMap ④:动态指定条件组获取信息:@QueryMap
\label{sec-1-2-1-4}
\begin{minted}[frame=lines,fontsize=\scriptsize,linenos=false]{java}
@GET("group/{id}/users")
Call<List<User>> groupList(@Path("id") int groupId, @QueryMap Map<String, String> options);
\end{minted}
\begin{itemize}
\item 使用@QueryMap注解可以分别地从Map集合中获取到元素,然后进行逐个的拼接在一起。
\item ok,到这里,我们使用@Get注解已经可以完成绝大部分的查询任务了,下面我们再来看看另一种常用的请求方式–post
\end{itemize}
\end{enumerate}
\subsubsection{2 POST : 一种用于携带传输数据的请求方式}
\label{sec-1-2-2}
\begin{itemize}
\item 稍微了解点Http的同学们,可能都会知道:相对于get请求方式把数据存放在uri地址栏中,post请求传输的数据时存放在请求体中,所以post才能做到对数据的大小无限制。而在Retrofit中,它又是怎么使用的呢?请看下面:
\end{itemize}
\begin{enumerate}
\item @Body ①:携带数据类型为对象时:@Body
\label{sec-1-2-2-1}
\begin{minted}[frame=lines,fontsize=\scriptsize,linenos=false]{java}
@POST("users/new")
Call<User> createUser(@Body User user);
\end{minted}
\begin{itemize}
\item 当我们的请求数据为某对象时Retrofit是这么处理使用的:
\begin{itemize}
\item 首先,Retrofit用@POST注解,标明这个是post的请求方式,里面是请求的url;
\item 其次,Retrofit仿照http直接提供了@Body注解,也就类似于直接把我们要传输的数据放在了body请求体中,这样应用可以更好的方便我们理解。
\end{itemize}
\item 来看下应用:
\end{itemize}
\begin{minted}[frame=lines,fontsize=\scriptsize,linenos=false]{java}
Call<List<User>> repos = service.createUser(new User(1, "管满满", "28", "http://write.blog.csdn.net/postlist"));
\end{minted}
\begin{itemize}
\item 这样我们直接把一个新的User对象利用注解@Body存放在body请求体,并随着请求的执行传输过去了。
\item 但是有同学在这该有疑问了,Retrofit就只能传输的数据为对象吗?当然不是,下面请看
\end{itemize}
\item @Field ②:携带数据类型为表单键值对时:@Field
\label{sec-1-2-2-2}
\begin{minted}[frame=lines,fontsize=\scriptsize,linenos=false]{java}
@FormUrlEncoded
@POST("user/edit")
Call<User> updateUser(@Field("first_name") String first, @Field("last_name") String last);
\end{minted}
\begin{itemize}
\item 当我们要携带的请求数据为表单时,通常会以键值对的方式呈现,那么Retrofit也为我们考虑了这种情况,它首先用到@FormUrlEncoded注解来标明这是一个表单请求,然后在我们的请求方法中使用@Field注解来标示所对应的String类型数据的键,从而组成一组键值对进行传递。
\item 那你是不是有该有疑问了,假如我是要上传一个文件呢?
\end{itemize}
\item @Part ③:单文件上传时:@Part
\label{sec-1-2-2-3}
\begin{minted}[frame=lines,fontsize=\scriptsize,linenos=false]{java}
@Multipart // 表示允许多个@Part
@PUT("user/photo")
Call<User> updateUser(@Part("photo") RequestBody photo, @Part("description") RequestBody description);
\end{minted}
\begin{itemize}
\item 此时在上传文件时,我们需要用@Multipart注解注明,它表示允许多个@Part,@Part则对应的一个RequestBody 对象,RequestBody 则是一个多类型的,当然也是包括文件的。下面看看使用
\end{itemize}
\begin{minted}[frame=lines,fontsize=\scriptsize,linenos=false]{java}
File file = new File(Environment.getExternalStorageDirectory(), "ic_launcher.png");
RequestBody photoRequestBody = RequestBody.create(MediaType.parse("image/png"), file);
RequestBody descriptionRequestBody = RequestBody.create(null, "this is photo.");
Call<User> call = service.updateUser(photoRequestBody, descriptionRequestBody);
\end{minted}
\begin{itemize}
\item 这里我们创建了两个RequestBody 对象,然后调用我们定义的updateUser方法,并把RequestBody传递进入,这样就实现了文件的上传。是不是很简单呢?
\item 相比单文件上传,Retrofit还进一步提供了多文件上传的方式:
\end{itemize}
\item @PartMap ④:多文件上传时:@PartMap
\label{sec-1-2-2-4}
\begin{minted}[frame=lines,fontsize=\scriptsize,linenos=false]{java}
@Multipart
@PUT("user/photo")
Call<User> updateUser(@PartMap Map<String, RequestBody> photos, @Part("description") RequestBody description);
\end{minted}
\begin{itemize}
\item 这里其实和单文件上传是差不多的,只是使用一个集合类型的Map封装了文件,并用@PartMap注解来标示起来。其他的都一样,这里就不多讲了。
\end{itemize}
\end{enumerate}
\subsubsection{3 Header : 一种用于携带消息头的请求方式}
\label{sec-1-2-3}
\begin{itemize}
\item Http请求中,为了防止攻击或是过滤掉不安全的访问或是为添加特殊加密的访问等等以减轻服务器的压力和保证请求的安全,通常都会在消息头中携带一些特殊的消息头处理。Retrofit也为我们提供了该请求方式:
\end{itemize}
\begin{minted}[frame=lines,fontsize=\scriptsize,linenos=false]{java}
@Headers("Cache-Control: max-age=640000")
@GET("widget/list")
Call<List<Widget>> widgetList();
// -----------------------------------------------------------
@Headers({
    "Accept: application/vnd.github.v3.full+json",
    "User-Agent: Retrofit-Sample-App"
})
@GET("users/{username}")
Call<User> getUser(@Path("username") String username);
\end{minted}
\begin{itemize}
\item 以上两种是静态的为Http请求添加消息头,只需要使用@Headers注解,以键值对的方式存放即可,如果需要添加多个消息头,则使用\{\}包含起来,如上所示。但要注意,即使有相同名字得消息头也不会被覆盖,并共同的存放在消息头中。
\item 当然有静态添加那相对的也就有动态的添加消息头了,方法如下:
\end{itemize}
\begin{minted}[frame=lines,fontsize=\scriptsize,linenos=false]{java}
@GET("user")
Call<User> getUser(@Header("Authorization") String authorization)
\end{minted}
\begin{itemize}
\item 使用@Header注解可以为一个请求动态的添加消息头,假如@Header对应的消息头为空的话,则会被忽略,否则会以它的.toString()方式输出。
\item ok,到这里已基本讲解完Retrofit的使用,还有两个重要但简单的方法也必须在这里提一下:
\begin{itemize}
\item 1 call.cancel();它可以终止正在进行的请求,程序只要一旦调用到它,不管请求是否在终止都会被停止掉。
\item 2 call.clone();当你想要多次请求一个接口的时候,直接用 clone 的方法来生产一个新的,否则将会报错,因为当你得到一个call实例,我们调用它的 execute 方法,但是这个方法只能调用一次。多次调用则发生异常。
\end{itemize}
\item 好了,关于Retrofit的使用我们就讲这么多,接下来我们从源码的角度简单的解析下它的实现原理。
\end{itemize}

\subsection{Retrofit2 从源码解析实现原理}
\label{sec-1-3}
\begin{itemize}
\item 首先先看一下Retrofit2标准示例
\end{itemize}
\begin{minted}[frame=lines,fontsize=\scriptsize,linenos=false]{java}
Retrofit retrofit = new Retrofit.Builder()
        .baseUrl(BASE_URL)
        .addConverterFactory(GsonConverterFactory.create())
        .build();
GitHubService service = retrofit.create(GitHubService.class);
service.enqueue();
\end{minted}
\begin{itemize}
\item 由上面我们基本可以看出,Retrofit是通过构造者模式创建出来的,那么我们就来看看Builder这个构造器的源码:
\end{itemize}
\begin{minted}[frame=lines,fontsize=\scriptsize,linenos=false]{java}
public static final class Builder {
    Builder(Platform platform) {
        this.platform = platform;
        converterFactories.add(new BuiltInConverters());
    }
    public Builder() {
        this(Platform.get());
    }

    public Builder client(OkHttpClient client) {
        return callFactory(checkNotNull(client, "client == null"));
    }
    public Builder callFactory(okhttp3.Call.Factory factory) {
        this.callFactory = checkNotNull(factory, "factory == null");
        return this;
    }

    public Builder baseUrl(String baseUrl) {
        checkNotNull(baseUrl, "baseUrl == null");
        HttpUrl httpUrl = HttpUrl.parse(baseUrl);
        if (httpUrl == null) 
            throw new IllegalArgumentException("Illegal URL: " + baseUrl);
        return baseUrl(httpUrl);
    }
    public Builder baseUrl(HttpUrl baseUrl) {
        checkNotNull(baseUrl, "baseUrl == null");
        List<String> pathSegments = baseUrl.pathSegments();
        if (!"".equals(pathSegments.get(pathSegments.size() - 1))) 
            throw new IllegalArgumentException("baseUrl must end in /: " + baseUrl);
        this.baseUrl = baseUrl;
        return this;
    }

    public Builder addConverterFactory(Converter.Factory factory) {
        converterFactories.add(checkNotNull(factory, "factory == null"));
        return this;
    }
    public Builder addCallAdapterFactory(CallAdapter.Factory factory) {
        adapterFactories.add(checkNotNull(factory, "factory == null"));
        return this;
    }

    public Builder callbackExecutor(Executor executor) {
        this.callbackExecutor = checkNotNull(executor, "executor == null");
        return this;
    }
    public Builder validateEagerly(boolean validateEagerly) {
        this.validateEagerly = validateEagerly;
        return this;
    }
    public Retrofit build() {
        if (baseUrl == null) 
            throw new IllegalStateException("Base URL required.");

        okhttp3.Call.Factory callFactory = this.callFactory;
        if (callFactory == null) 
            callFactory = new OkHttpClient();

        Executor callbackExecutor = this.callbackExecutor;
        if (callbackExecutor == null) 
            callbackExecutor = platform.defaultCallbackExecutor();

        List<CallAdapter.Factory> adapterFactories = new ArrayList<>(this.adapterFactories);
        adapterFactories.add(platform.defaultCallAdapterFactory(callbackExecutor));
        List<Converter.Factory> converterFactories = new ArrayList<>(this.converterFactories);

        return new Retrofit(callFactory, baseUrl, converterFactories, adapterFactories, callbackExecutor, validateEagerly);
    }
}
\end{minted}
\begin{itemize}
\item 源码讲解:
\begin{itemize}
\item 1:当我们使用new Retrofit.Builder()来创建时,在Builder构造器中,首先就获得当前的设备平台信息,并且把内置的转换器工厂(BuiltInConverters)加添到工厂集合中,它的主要作用就是当使用多种Converters的时候能够正确的引导并找到可以消耗该类型的转化器。
\item 2:从我们的基本示例中看到有调用到.baseUrl(BASE$_{\text{URL}}$)这个方法,实际上没当使用Retrofit时,该方法都是必须传入的,并且还不能为空,从源码中可以看出,当baseUrl方法传进的参数来看,如果为空的话将会抛出NullPointerException空指针异常。
\item 3:addConverterFactory该方法是传入一个转换器工厂,它主要是对数据转化用的,请网络请求获取的数据,将会在这里被转化成我们所需要的数据类型,比如通过Gson将json数据转化成对象类型。
\item 4 : 从源码中,我们看到还有一个client方法,这个是可选的,如果没有传入则就默认为OkHttpClient,在这里可以对OkHttpClient做一些操作,比如添加拦截器打印log等
\item 5:callbackExecutor该方法从名字上看可以得知应该是回调执行者,也就是Call对象从网络服务获取数据之后转换到UI主线程中。
\item 6:addCallAdapterFactory该方法主要是针对Call转换了,比如对Rxjava的支持,从返回的call对象转化为Observable对象。
\item 7:最后调用build()方法,通过new Retrofit(callFactory, baseUrl, converterFactories, adapterFactories, callbackExecutor, validateEagerly);构造方法把所需要的对象传递到Retrofit对象中。
\end{itemize}
\item ok,当我们通过Builder构造器构造出Retrofit对象时,然后通过Retrofit.create()方法是怎么把我们所定义的接口转化成接口实例的呢?来看下create()源码:
\end{itemize}
\begin{minted}[frame=lines,fontsize=\scriptsize,linenos=false]{java}
public <T> T create(final Class<T> service) {
    Utils.validateServiceInterface(service);
    if (validateEagerly) 
        eagerlyValidateMethods(service);
    return (T) Proxy.newProxyInstance(service.getClassLoader(), new Class<?>[] { service }, new InvocationHandler() {
            private final Platform platform = Platform.get();
            @Override public Object invoke(Object proxy, Method method, Object... args) throws Throwable {
                // If the method is a method from Object then defer to normal invocation.
                if (method.getDeclaringClass() == Object.class) 
                    return method.invoke(this, args);
                if (platform.isDefaultMethod(method)) 
                    return platform.invokeDefaultMethod(method, service, proxy, args);
                ServiceMethod serviceMethod = loadServiceMethod(method);
                OkHttpCall okHttpCall = new OkHttpCall<>(serviceMethod, args);
                return serviceMethod.callAdapter.adapt(okHttpCall);
            }
        });
}
\end{minted}
\begin{itemize}
\item 当看到Proxy时,是不是多少有点明悟了呢?没错就是动态代理,动态代理其实已经封装的很简单了,主要使用newProxyInstance()方法来返回一个类的代理实例,其中它内部需要传递一个类的加载器,类本身以及一个InvocationHandler处理器,主要的动作都是在InvocationHandler中进行的,它里面只有一个方法invoke()方法,每当我们调用代理类里面的方法时invoke()都会被执行,并且我们可以从该方法的参数中获取到所需要的一切信息,比如从method中获取到方法名,从args中获取到方法名中的参数信息等。
\item 而Retrofit在这里使用到动态代理也不会例外:
\begin{itemize}
\item 首先,通过method把它转换成ServiceMethod ;
\item 然后,通过serviceMethod, args获取到okHttpCall 对象;
\item 最后,再把okHttpCall进一步封装并返回Call对象。
\end{itemize}
\item 下面来逐步详解。
\end{itemize}
\subsubsection{1:将method把它转换成ServiceMethod}
\label{sec-1-3-1}
\begin{minted}[frame=lines,fontsize=\scriptsize,linenos=false]{java}
ServiceMethod serviceMethod = loadServiceMethod(method);
ServiceMethod loadServiceMethod(Method method) {
    ServiceMethod result;
    synchronized (serviceMethodCache) {
        result = serviceMethodCache.get(method);
        if (result == null) {
            result = new ServiceMethod.Builder(this, method).build();
            serviceMethodCache.put(method, result);
        }
    }
    return result;
}
\end{minted}
\begin{itemize}
\item loadServiceMethod源码方法中非常的好理解,主要就是通过ServiceMethod.Builder()方法来构建ServiceMethod,并把它给缓存取来,以便下次可以直接回去ServiceMethod。那下面我们再来看看它是怎么构建ServiceMethod方法的:
\end{itemize}
\begin{minted}[frame=lines,fontsize=\scriptsize,linenos=false]{java}
public Builder(Retrofit retrofit, Method method) {
    this.retrofit = retrofit;
    this.method = method;
    this.methodAnnotations = method.getAnnotations();
    this.parameterTypes = method.getGenericParameterTypes();
    this.parameterAnnotationsArray = method.getParameterAnnotations();
}
public ServiceMethod build() {
    callAdapter = createCallAdapter();
    responseType = callAdapter.responseType();
    if (responseType == Response.class || responseType == okhttp3.Response.class) 
        throw methodError("'" + Utils.getRawType(responseType).getName()
                          + "' is not a valid response body type. Did you mean ResponseBody?");
    responseConverter = createResponseConverter();
    return new ServiceMethod<>(this);
}
\end{minted}
\begin{itemize}
\item 首先在Builder()中初始化一些参数,然后在build()中返回一个new ServiceMethod<>(this)对象。
\item 下面来详细的解释下build()方法,完全理解了该方法则便于理解下面的所有执行流程。
\end{itemize}
\begin{enumerate}
\item ①:构建CallAdapter对象,该对象将会在第三步中起着至关重要的作用。
\label{sec-1-3-1-1}
\begin{itemize}
\item 现在我们先看看它是怎么构建CallAdapter对象的:createCallAdapter()方法源码如下:
\end{itemize}
\begin{minted}[frame=lines,fontsize=\scriptsize,linenos=false]{java}
private CallAdapter<?> createCallAdapter() {
    Type returnType = method.getGenericReturnType();
    if (Utils.hasUnresolvableType(returnType)) 
        throw methodError("Method return type must not include a type variable or wildcard: %s", returnType);
    if (returnType == void.class) 
        throw methodError("Service methods cannot return void.");
    Annotation[] annotations = method.getAnnotations();
    try {
        return retrofit.callAdapter(returnType, annotations);
    } catch (RuntimeException e) { // Wide exception range because factories are user code.
        throw methodError(e, "Unable to create call adapter for %s", returnType);
    }
}
\end{minted}
\begin{itemize}
\item 在createCallAdapter方法中主要做的是事情就是获取到method的类型和注解,然后调用retrofit.callAdapter(returnType, annotations);方法:
\end{itemize}
\begin{minted}[frame=lines,fontsize=\scriptsize,linenos=false]{java}
public CallAdapter<?> callAdapter(Type returnType, Annotation[] annotations) {
    return nextCallAdapter(null, returnType, annotations);
}
public CallAdapter<?> nextCallAdapter(CallAdapter.Factory skipPast, Type returnType, Annotation[] annotations) {
    checkNotNull(returnType, "returnType == null");
    checkNotNull(annotations, "annotations == null");
    int start = adapterFactories.indexOf(skipPast) + 1;
    for (int i = start, count = adapterFactories.size(); i < count; i++) {
        CallAdapter<?> adapter = adapterFactories.get(i).get(returnType, annotations, this);
        if (adapter != null) 
            return adapter;
    }
}
\end{minted}
\begin{itemize}
\item 辗转到Retrofit中nextCallAdapter()中,在for 循环中分别从adapterFactories中来获取CallAdapter对象,但是adapterFactories中有哪些CallAdapter对象呢,这就需要返回到构建Retrofit对象中的Builder 构造器中查看了
\end{itemize}
\begin{minted}[frame=lines,fontsize=\scriptsize,linenos=false]{java}
public static final class Builder {
    public Builder addCallAdapterFactory(CallAdapter.Factory factory) {
        adapterFactories.add(checkNotNull(factory, "factory == null"));
        return this;
    }
    public Retrofit build() {
        List<CallAdapter.Factory> adapterFactories = new ArrayList<>(this.adapterFactories);
        adapterFactories.add(platform.defaultCallAdapterFactory(callbackExecutor));
    }  
}
CallAdapter.Factory defaultCallAdapterFactory(Executor callbackExecutor) {
    if (callbackExecutor != null) 
        return new ExecutorCallAdapterFactory(callbackExecutor);
    return DefaultCallAdapterFactory.INSTANCE;
}
\end{minted}
\begin{itemize}
\item 从上面的代码中可以看到,不管有没有通过addCallAdapterFactory添加CallAdapter,adapterFactories集合至少都会有一个ExecutorCallAdapterFactory对象。当我们从adapterFactories集合中回去CallAdapter对象时,那我们都会获得ExecutorCallAdapterFactory这个对象。而这个对象将在第三步中和后面执行同步或异步请求时起着至关重要的作用。
\end{itemize}
\item ②:构建responseConverter转换器对象,它的作用是寻找适合的数据类型转化
\label{sec-1-3-1-2}
\begin{itemize}
\item 该对象的构建和构建CallAdapter对象的流程基本是一致的,这里就不在赘述。同学们可自行查看源码。
\end{itemize}
\end{enumerate}
\subsubsection{2:通过serviceMethod, args获取到okHttpCall 对象}
\label{sec-1-3-2}
\begin{itemize}
\item 第二步相对比较简单,就是对象传递:
\end{itemize}
\begin{minted}[frame=lines,fontsize=\scriptsize,linenos=false]{java}
OkHttpCall(ServiceMethod<T> serviceMethod, Object[] args) {
    this.serviceMethod = serviceMethod;
    this.args = args;
}
\end{minted}
\subsubsection{3:把okHttpCall进一步封装并返回Call对象}
\label{sec-1-3-3}
\begin{itemize}
\item 这一步也是一句话 return serviceMethod.callAdapter.adapt(okHttpCall);但是想理解清楚必须先把第一步理解透彻,通过第一步我们找得到serviceMethod.callAdapter就是ExecutorCallAdapterFactory对象,那么调用.adapt(okHttpCall)把okHttpCall怎么进行封装呢?看看源码:
\end{itemize}
\begin{minted}[frame=lines,fontsize=\scriptsize,linenos=false]{java}
T adapt(Call call);
\end{minted}
\begin{itemize}
\item 一看,吓死宝宝了,就这么一句,这是嘛呀,但是经过第一步的分析,我们已知道serviceMethod.callAdapter就是ExecutorCallAdapterFactory,那么我们可以看看在ExecutorCallAdapterFactory类中有没有发现CallAdapter的另类应用呢,一看,果不其然在重写父类的get()方法中我们找到了答案:
\end{itemize}
\begin{minted}[frame=lines,fontsize=\scriptsize,linenos=false]{java}
@Override
public CallAdapter<Call<?>> get(Type returnType, Annotation[] annotations, Retrofit retrofit) {
    if (getRawType(returnType) != Call.class) 
        return null;
    final Type responseType = Utils.getCallResponseType(returnType);
    return new CallAdapter<Call<?>>() {
        @Override public Type responseType() {
            return responseType;
        }
        @Override public <R> Call<R> adapt(Call<R> call) {
            return new ExecutorCallbackCall<>(callbackExecutor, call);
        }
    };
}
\end{minted}
\begin{itemize}
\item 当看到return new CallAdapter中的adapt(Call call)我们就完全知其所以然了,至于ExecutorCallbackCall怎么应用的我们在发起网络请求的时候讲解。
\item ok,当我们得到接口的代理实例之后,通过代理接口调用里面的方法,就会触发InvocationHandler对象中的invoke方法,从而完成上面的三个步骤并且返回一个Call对象,通过Call对象就可以去完成我们的请求了,Retrofit为我们提供两种请求方式,一种是同步,一种是异步。我们这里就以异步方式来讲解:
\end{itemize}
\begin{minted}[frame=lines,fontsize=\scriptsize,linenos=false]{java}
service.enqueue(new Callback<List<User>>() {
    @Override
    public void onResponse(Call<List<User>> call, Response<List<User>> response) {
        Log.d("response body",response.body());
    }
    @Override
    public void onFailure(Call<BoardInfo> call, Throwable t) {
        Log.i("response Throwable",t.getMessage().toString());
    }
});
\end{minted}
\begin{itemize}
\item 从上面我们可以看到enqueue方法中有一个回调函数,回调函数里面重写了两个方法分别代表请求成功和失败的方法,但是我们想知道它是怎么实现的原理呢?那么请往下面看:
\item 在上面获取接口的代理实例时,通过代理接口调用里面的方法获取一个Call对象,我们上面也分析了其实这个Call对象就是ExecutorCallbackCall,那么我们来看看它里面是怎么实现的?
\end{itemize}
\begin{minted}[frame=lines,fontsize=\scriptsize,linenos=false]{java}
static final class ExecutorCallbackCall<T> implements Call<T> {
    final Executor callbackExecutor;
    final Call<T> delegate;
    ExecutorCallbackCall(Executor callbackExecutor, Call<T> delegate) {
        this.callbackExecutor = callbackExecutor;
        this.delegate = delegate;
    }
    @Override public void enqueue(final Callback<T> callback) {
        if (callback == null) throw new NullPointerException("callback == null");

        delegate.enqueue(new Callback<T>() {
                @Override public void onResponse(Call<T> call, final Response<T> response) {
                    callbackExecutor.execute(new Runnable() {
                            @Override public void run() {
                                if (delegate.isCanceled()) 
                                    callback.onFailure(ExecutorCallbackCall.this, new IOException("Canceled"));
                                else 
                                    callback.onResponse(ExecutorCallbackCall.this, response);
                            }
                        });
                }
                @Override public void onFailure(Call<T> call, final Throwable t) {
                    callbackExecutor.execute(new Runnable() {
                            @Override public void run() {
                                callback.onFailure(ExecutorCallbackCall.this, t);
                            }
                        });
                }
            });
    }
}
\end{minted}
\begin{itemize}
\item 在ExecutorCallbackCall类中,封装了两个对象一个是callbackExecutor,它主要是把现在运行的线程切换到主线程中去,一个是delegate对象,这个对象就是真真正正的执行网络操作的对象,那么它的真身到底是什么呢?还记得我们在获取代理接口第三步执行的serviceMethod.callAdapter.adapt(okHttpCall)的分析吧,经过辗转几步终于把okHttpCall传递到了new ExecutorCallbackCall<>(callbackExecutor, call);中,然后看看ExecutorCallbackCall的构造方法:
\end{itemize}
\begin{minted}[frame=lines,fontsize=\scriptsize,linenos=false]{java}
ExecutorCallbackCall(Executor callbackExecutor, Call<T> delegate) {
      this.callbackExecutor = callbackExecutor;
      this.delegate = delegate;
    }
\end{minted}
\begin{itemize}
\item 由此可以明白delegate 就是okHttpCall对象,那么我们在看看okHttpCall是怎么执行异步网络请求的:
\end{itemize}
\begin{minted}[frame=lines,fontsize=\scriptsize,linenos=false]{java}
@Override 
public void enqueue(final Callback<T> callback) {
    if (callback == null) throw new NullPointerException("callback == null");
    okhttp3.Call call;
    Throwable failure;
    synchronized (this) {
        if (executed) throw new IllegalStateException("Already executed.");
        executed = true;
        call = rawCall;
        failure = creationFailure;
        if (call == null && failure == null) 
            try {
                call = rawCall = createRawCall();
            } catch (Throwable t) {
                failure = creationFailure = t;
            }
    }
    if (failure != null) {
        callback.onFailure(this, failure);
        return;
    }
    if (canceled) 
        call.cancel();
    call.enqueue(new okhttp3.Callback() {
            @Override public void onResponse(okhttp3.Call call, okhttp3.Response rawResponse)
                throws IOException {
                Response<T> response;
                try {
                    response = parseResponse(rawResponse);
                } catch (Throwable e) {
                    callFailure(e);
                    return;
                }
                callSuccess(response);
            }
            @Override public void onFailure(okhttp3.Call call, IOException e) {
                try {
                    callback.onFailure(OkHttpCall.this, e);
                } catch (Throwable t) {
                    t.printStackTrace();
                }
            }
            private void callFailure(Throwable e) {
                try {
                    callback.onFailure(OkHttpCall.this, e);
                } catch (Throwable t) {
                    t.printStackTrace();
                }
            }
            private void callSuccess(Response<T> response) {
                try {
                    callback.onResponse(OkHttpCall.this, response);
                } catch (Throwable t) {
                    t.printStackTrace();
                }
            }
        });
}
\end{minted}
\begin{itemize}
\item 从上面代码中,我们很容易就看出,其实它就是这里面封装了一个okhttp3.Call,直接利用okhttp进行网络的异步操作,至于okhttp是怎么进行网络请求的我们就不再这里讲解了,感兴趣的朋友可以自己去查看源码。
\end{itemize}

\section{Retrofit库的核心实现原理是什么?如果让你实现这个库的某些核心功能,你会考虑怎么去实现?}
\label{sec-2}
\begin{itemize}
\item Retrofit主要是在create方法中采用动态代理模式(通过访问代理对象的方式来间接访问目标对象)实现接口方法,这个过程构建了一个ServiceMethod对象,根据方法注解获取请求方式,参数类型和参数注解拼接请求的链接,当一切都准备好之后会把数据添加到Retrofit的RequestBuilder中。然后当我们主动发起网络请求的时候会调用okhttp发起网络请求,okhttp的配置包括请求方式,URL等在Retrofit 的RequestBuilder的build()方法中实现,并发起真正的网络请求。
\item 你从这个库中学到什么有价值的或者说可借鉴的设计思想?
\item 内部使用了优秀的架构设计和大量的设计模式,在我分析过Retrofit最新版的源码和大量优秀的Retrofit 源码分析文章后,我发现,要想真正理解Retrofit内部的核心源码流程和设计思想,首先,需要对它使用到的九大设计模式有一定的了解,下面我简单说一说:
\item 1、创建Retrofit实例: 使用 \uline{建造者模式} 通过内部Builder类建立了一个Retroift实例。 网络请求工厂使用了 \uline{工厂模式} 方法。
\item 2、创建网络请求接口的实例:
\begin{itemize}
\item 首先,使用 \uline{外观模式} 统一调用创建网络请求接口实例和网络请求参数配置的方法。 然后,使用 \uline{动态代理} 动态地去创建网络请求接口实例。
\item 接着,使用了 \uline{建造者模式} \& \uline{单例模式} 创建了serviceMethod对象。
\item 再者,使用了 \uline{策略模式} 对serviceMethod对象进行网络请求参数配置,即通过解析网络请求接口方 法的参数、返回值和注解类型,从Retrofit对象中获取对应的网络的url地址、网络请求执行器、网 络请求适配器和数据转换器。
\item 最后,使用了 \uline{装饰者模式} ExecuteCallBack为serviceMethod对象加入线程切换的操作,便于接受 数据后通过Handler从子线程切换到主线程从而对返回数据结果进行处理。
\end{itemize}
\item 3、发送网络请求: 在异步请求时,通过静态delegate代理对网络请求接口的方法中的每个参数使用对应的ParameterHanlder进行解析。
\item 4、解析数据
\item 5、切换线程: 使用了 \uline{适配器模式} 通过检测不同的Platform使用不同的回调执行器,然后使用回调执行器切换线程,这里同样是使用了装饰模式。
\item 6、处理结果
\end{itemize}
% Emacs 27.1 (Org mode 8.2.7c)
\end{document}
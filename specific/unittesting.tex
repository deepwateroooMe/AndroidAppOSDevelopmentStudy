% Created 2021-12-21 Tue 15:51
\documentclass[9pt, b5paper]{article}
\usepackage[UTF8]{ctex}
\usepackage{xltxtra}
\usepackage{bera}
\usepackage[T1]{fontenc}
\usepackage[scaled]{beraserif}
\usepackage[scaled]{berasans}
\usepackage[scaled]{beramono}
\usepackage{graphicx}
\usepackage{xcolor}
\usepackage{multirow}
\usepackage{multicol}
\usepackage{float}
\usepackage{textcomp}
\usepackage{geometry}
\geometry{left=1.2cm,right=1.2cm,top=1.5cm,bottom=1.2cm}
\usepackage{algorithm}
\usepackage{algorithmic}
\usepackage{latexsym}
\usepackage{natbib}
\usepackage{minted}
\newminted{common-lisp}{fontsize=ootnotesize}
\usepackage[xetex,colorlinks=true,CJKbookmarks=true,linkcolor=blue,urlcolor=blue,menucolor=blue]{hyperref}
\author{deepwaterooo}
\date{\today}
\title{JUnit Mockito unit testing}
\hypersetup{
  pdfkeywords={},
  pdfsubject={},
  pdfcreator={Emacs 27.1 (Org mode 8.2.7c)}}
\begin{document}

\maketitle
\tableofcontents


\section{Mockito (AndroidUT)}
\label{sec-1}
\begin{itemize}
\item \url{https://blog.csdn.net/qq_17766199/article/details/78450007}
\item 添加Mockito的依赖:
\end{itemize}
\begin{minted}[frame=lines,fontsize=\scriptsize,linenos=false]{groovy}
//mockito
testCompile "org.mockito:mockito-core:2.8.9"
androidTestCompile "org.mockito:mockito-android:2.8.9"
\end{minted}
\subsection{基本步骤}
\label{sec-1-1}
\subsubsection{初始化注入}
\label{sec-1-1-1}
\begin{itemize}
\item 首先我们在setUp函数中进行初始化:
\end{itemize}
\begin{minted}[frame=lines,fontsize=\scriptsize,linenos=false]{java}
private ArrayList mockList;
@Before
public void setUp() throws Exception {
    //MockitoAnnotations.initMocks(this);
    //mock creation
    mockList = mock(ArrayList.class);
}
\end{minted}
\begin{itemize}
\item 当然,你也这样这样进行注入:
\end{itemize}
\begin{minted}[frame=lines,fontsize=\scriptsize,linenos=false]{java}
@Mock
private ArrayList mockList;
@Before
public void setUp() throws Exception {
    MockitoAnnotations.initMocks(this); // 在base class中或者初始化的地方配置
}
\end{minted}
\begin{itemize}
\item initMocks(this)后,就可以通过@Mock注解直接使用mock对象。
\item 使用JUnit4的rule来配置:
\end{itemize}
\begin{minted}[frame=lines,fontsize=\scriptsize,linenos=false]{java}
@Rule
public MockitoRule rule = MockitoJUnit.rule().strictness(Strictness.STRICT_STUBS);
\end{minted}
\begin{itemize}
\item 以上2种方式可以达到同样的效果。

\item 下面列几种常用又比较典型的用法
\item 一个最普通的例子
\end{itemize}
\begin{minted}[frame=lines,fontsize=\scriptsize,linenos=false]{java}
  @Test
    public void sampleTest1() throws Exception {
        //使用mock对象执行方法
        mockList.add("one");
        mockList.clear();

        //检验方法是否调用
        verify(mockList).add("one");   
        verify(mockList).clear();     
    }
\end{minted}
\begin{itemize}
\item 我们可以直接调用mock对象的方法,比如ArrayList.add()或者ArrayList.clear(),然后我们通过verify函数进行校验。
\end{itemize}

\subsubsection{直接mock接口对象}
\label{sec-1-1-2}
\begin{itemize}
\item 正常来讲我们想要一个接口类型的对象,首先我们需要先实例化一个对象并实现,其对应的抽象方法,但是有了mock,我们可以直接mock出一个接口对象:
\end{itemize}
\begin{minted}[frame=lines,fontsize=\scriptsize,linenos=false]{java}
@Test
public void sampleTest2() throws Exception {
    //我们可以直接mock一个借口,即使我们并未声明它
    MVPContract.Presenter mockPresenter = mock(MVPContract.Presenter.class);
    when(mockPresenter.getUserName()).thenReturn("qingmei2"); //我们定义,当mockPresenter调用getUserName()时,返回qingmei2
    String userName = mockPresenter.getUserName();

    verify(mockPresenter).getUserName(); //校验 是否mockPresenter调用了getUserName()方法
    Assert.assertEquals("qingmei2", userName); //断言 userName为qingmei2

//        verify(mockPresenter).getPassword();  //校验 是否mockPresenter调用了getPassword()方法
    String password = mockPresenter.getPassword();  //因为未定义返回值,默认返回null
    verify(mockPresenter).getPassword();
    Assert.assertEquals(password, null);
}
\end{minted}
\subsubsection{参数匹配器}
\label{sec-1-1-3}
\begin{minted}[frame=lines,fontsize=\scriptsize,linenos=false]{java}
@Test
public void argumentMatchersTest3() throws Exception {
    when(mockList.get(anyInt())).thenReturn("不管请求第几个参数 我都返回这句");
    System.out.println(mockList.get(0));
    System.out.println(mockList.get(39));

    //当mockList调用addAll()方法时,「匹配器」如果传入的参数list size==2,返回true;
    when(mockList.addAll(argThat(getListMatcher()))).thenReturn(true);

    //根据API文档,我们也可以使用lambda表达式: 「匹配器」如果传入的参数list size==3,返回true;
//        when(mockList.addAll(argThat(list -> list.size() == 3))).thenReturn(true);
    //我们不要使用太严格的参数Matcher,也许下面会更好
//        when(mockList.addAll(argThat(notNull()));
    boolean b1 = mockList.addAll(Arrays.asList("one", "two"));
    boolean b2 = mockList.addAll(Arrays.asList("one", "two", "three"));
    verify(mockList).addAll(argThat(getListMatcher()));
    Assert.assertTrue(b1);
    Assert.assertTrue(!b2);
}
private ListOfTwoElements getListMatcher() {
    return new ListOfTwoElements();
}
/**
 * 匹配器,用来测试list是否有且仅存在两个元素
 */
class ListOfTwoElements implements ArgumentMatcher<List> {
    public boolean matches(List list) {
        return list.size() == 2;
    }

    public String toString() {
        //printed in verification errors
        return "[list of 2 elements]";
    }
}
\end{minted}
\begin{itemize}
\item 对于一个Mock的对象,有时我们需要进行校验,但是基础的API并不能满足我们校验的需要,我们可以自定义Matcher,比如案例中,我们自定义一个Matcher,只有容器中两个元素时,才会校验通过。
\end{itemize}
\subsubsection{验证方法的调用次数}
\label{sec-1-1-4}
\begin{minted}[frame=lines,fontsize=\scriptsize,linenos=false]{java}
 /**
   * 我们也可以测试方法调用的次数
   * https://static.javadoc.io/org.mockito/mockito-core/2.8.9/org/mockito/Mockito.html#exact_verification
   *
   * @throws Exception
   */
@Test
public void simpleTest4() throws Exception {
    mockList.add("once");

    mockList.add("twice");
    mockList.add("twice");

    mockList.add("three times");
    mockList.add("three times");
    mockList.add("three times");

    verify(mockList).add("once");  //验证mockList.add("once")调用了一次 - times(1) is used by default
    verify(mockList, times(1)).add("once");//验证mockList.add("once")调用了一次

    //调用多次校验
    verify(mockList, times(2)).add("twice");
    verify(mockList, times(3)).add("three times");

    //从未调用校验
    verify(mockList, never()).add("four times");

    //至少、至多调用校验
    verify(mockList, atLeastOnce()).add("three times");
    verify(mockList, atMost(5)).add("three times");
//        verify(mockList, atLeast(2)).add("five times"); //这行代码不会通过
}
\end{minted}
\subsubsection{抛出异常}
\label{sec-1-1-5}
\begin{minted}[frame=lines,fontsize=\scriptsize,linenos=false]{java}
/**
  * 异常抛出测试
  * https://static.javadoc.io/org.mockito/mockito-core/2.8.9/org/mockito/Mockito.html#stubbing_with_exceptions
  */
@Test
public void throwTest5() {
    doThrow(new NullPointerException("throwTest5.抛出空指针异常")).when(mockList).clear();
    doThrow(new IllegalArgumentException("你的参数似乎有点问题")).when(mockList).add(anyInt());

    mockList.add("string");//这个不会抛出异常
    mockList.add(12);//抛出了异常,因为参数是Int
    mockList.clear();
}
\end{minted}
\begin{itemize}
\item 如案例所示,当mockList对象执行clear方法时,抛出空指针异常,当其执行add方法,且传入的参数类型为int时,抛出非法参数异常。
\end{itemize}
\subsubsection{校验方法执行顺序}
\label{sec-1-1-6}
\begin{minted}[frame=lines,fontsize=\scriptsize,linenos=false]{java}
/**
  * 验证执行执行顺序
  * https://static.javadoc.io/org.mockito/mockito-core/2.8.9/org/mockito/Mockito.html#in_order_verification
  *
  * @throws Exception
  */
@Test
public void orderTest6() throws Exception {
    List singleMock = mock(List.class);

    singleMock.add("first add");
    singleMock.add("second add");

    InOrder inOrder = inOrder(singleMock);

    //inOrder保证了方法的顺序执行
    inOrder.verify(singleMock).add("first add");
    inOrder.verify(singleMock).add("second add");

    List firstMock = mock(List.class);
    List secondMock = mock(List.class);

    firstMock.add("first add");
    secondMock.add("second add");

    InOrder inOrder1 = inOrder(firstMock, secondMock);

    //下列代码会确认是否firstmock优先secondMock执行add方法
    inOrder1.verify(firstMock).add("first add");
    inOrder1.verify(secondMock).add("second add");
}
\end{minted}
\begin{itemize}
\item 有时候我们需要校验方法执行顺序的先后,如案例所示,inOrder对象会判断方法执行顺序,如果顺序不对,该测试案例failed。
\end{itemize}
\subsubsection{确保mock对象从未进行过交互:verifyZeroInteractions \&\& verifyNoMoreInteractions}
\label{sec-1-1-7}
\begin{minted}[frame=lines,fontsize=\scriptsize,linenos=false]{java}
/**
  * 确保mock对象从未进行过交互
  * https://static.javadoc.io/org.mockito/mockito-core/2.8.9/org/mockito/Mockito.html#never_verification
  *
  * @throws Exception
  */
@Test
public void noInteractedTest7() throws Exception {
    List firstMock = mock(List.class);
    List secondMock = mock(List.class);
    List thirdMock = mock(List.class);

    firstMock.add("one");

    verify(firstMock).add("one");

    verify(firstMock, never()).add("two");

    firstMock.add(thirdMock);
    // 确保交互(interaction)操作不会执行在mock对象上
//        verifyZeroInteractions(firstMock); //test failed,因为firstMock和其他mock对象有交互
    verifyZeroInteractions(secondMock, thirdMock);   // 0交互: test pass
}
\end{minted}
\begin{itemize}
\item 另一个小例子
\end{itemize}
\begin{minted}[frame=lines,fontsize=\scriptsize,linenos=false]{java}
 @Test
    public void testMock6() {
        List list = mock(List.class);
        // 验证mock对象没有产生任何交互,也即没有任何方法调用
        verifyZeroInteractions(list);

        List list2 = mock(List.class);
        list2.add("one");
        list2.add("two");
        verify(list2).add("one");
        // 验证mock对象是否有被调用过但没被验证的方法。这里会测试不通过,list2.add("two")方法没有被验证过
        verifyNoMoreInteractions(list2); // 
    }
\end{minted}
\begin{itemize}
\item 可能是因为水平有限,笔者很少用到这个API(好吧除了学习案例中用过其他基本没怎么用过),不过还是敲一遍,保证有个基础的印象。
\end{itemize}
\subsubsection{简化mock对象的创建}
\label{sec-1-1-8}
\begin{minted}[frame=lines,fontsize=\scriptsize,linenos=false]{java}
/**
  * 简化mock对象的创建,请注意,一旦使用@Mock注解,一定要在测试方法调用之前调用(比如@Before注解的setUp方法)
  * MockitoAnnotations.initMocks(testClass);
  */
@Mock
List mockedList;
@Mock
User mockedUser;
@Test
public void initMockTest8() throws Exception {
    mockedList.add("123");
    mockedUser.setLogin("qingmei2");
}
\end{minted}
\begin{itemize}
\item 注释写的很明白了,不赘述
\end{itemize}
\subsubsection{重置mocks}
\label{sec-1-1-9}
\begin{minted}[frame=lines,fontsize=\scriptsize,linenos=false]{java}
@Test
public void testMock12() {
    List list = mock(List.class);
    when(list.size()).thenReturn(100);
    // 打印出"100"
    System.out.println(list.size());
    // 充值mock, 之前的交互和stub将全部失效
    reset(list);
    // 打印出"0"
    System.out.println(list.size());
}
\end{minted}
\subsubsection{使用 @InjectMocks 在 Mockito 中进行依赖注入}
\label{sec-1-1-10}
\begin{itemize}
\item 我们也可以使用@InjectMocks 注解来创建对象,它会根据类型来注入对象里面的成员方法和变量。假定我们有 ArticleManager 类:
\end{itemize}
\begin{minted}[frame=lines,fontsize=\scriptsize,linenos=false]{java}
public class ArticleManager {
    private ArticleDatabase database;
    private User user;
    public ArticleManager(User user, ArticleDatabase database) {
        super();
        this.user = user;
        this.database = database;
    }
    public void initialize() {
        database.addListener(new ArticleListener());
    }
}
\end{minted}
\begin{itemize}
\item 这个类可以通过Mockito构建,它的依赖可以通过模拟对象来实现,如下面的代码片段所示:
\end{itemize}
\begin{minted}[frame=lines,fontsize=\scriptsize,linenos=false]{java}
@RunWith(MockitoJUnitRunner.class)
public class ArticleManagerTest  {
       @Mock User user;
       @Mock ArticleDatabase database;
       @Mock ArticleCalculator calculator;
       @Spy private UserProvider userProvider = new ConsumerUserProvider();

       @InjectMocks private ArticleManager manager; // 

       @Test public void shouldDoSomething() {
           // calls addListener with an instance of ArticleListener
           manager.initialize();
           // validate that addListener was called
           verify(database).addListener(any(ArticleListener.class));
       }
}
\end{minted}
\begin{itemize}
\item 创建一个实例ArticleManager并将其注入到它中
\item 更多的详情可以查看:
\begin{itemize}
\item \url{http://docs.mockito.googlecode.com/hg/1.9.5/org/mockito/InjectMocks.html}
\end{itemize}
\end{itemize}

\subsubsection{方法连续调用测试}
\label{sec-1-1-11}
\begin{minted}[frame=lines,fontsize=\scriptsize,linenos=false]{java}
/**
  * 方法连续调用的测试
  * https://static.javadoc.io/org.mockito/mockito-core/2.8.9/org/mockito/Mockito.html#stubbing_consecutive_calls
  */
@Test
public void continueMethodTest9() throws Exception {
    when(mockedUser.getName())
            .thenReturn("qingmei2")
            .thenThrow(new RuntimeException("方法调用第二次抛出异常"))
            .thenReturn("qingemi2 第三次调用");

    //另外一种方式
    when(mockedUser.getName()).thenReturn("qingmei2 1", "qingmei2 2", "qingmei2 3");
    String name1 = mockedUser.getName();
    try {
        String name2 = mockedUser.getName();
    } catch (Exception e) {
        System.out.println(e.getMessage());
    }
    String name3 = mockedUser.getName();
    System.out.println(name1);
    System.out.println(name3);
}
\end{minted}
\begin{itemize}
\item 有用,但不重要,学习一下加深印象。
\end{itemize}
\subsubsection{为回调方法做测试}
\label{sec-1-1-12}
\begin{minted}[frame=lines,fontsize=\scriptsize,linenos=false]{java}
/**
  * 为回调方法做测试
  * https://static.javadoc.io/org.mockito/mockito-core/2.8.9/org/mockito/Mockito.html#answer_stubs
  */
@Test
public void callBackTest() throws Exception {
    when(mockList.add(anyString())).thenAnswer(new Answer<Boolean>() {
        @Override
        public Boolean answer(InvocationOnMock invocation) throws Throwable {
            Object[] args = invocation.getArguments();
            Object mock = invocation.getMock();
            return false;
        }
    });
    System.out.println(mockList.add("第1次返回false"));
    //lambda表达式
    when(mockList.add(anyString())).then(invocation -> true);
    System.out.println(mockList.add("第2次返回true"));

    when(mockList.add(anyString())).thenReturn(false);
    System.out.println(mockList.add("第3次返回false"));
}
\end{minted}
\begin{itemize}
\item 在Mockito的官方文档中,这样写道:
\begin{itemize}
\item 在最初的Mockito里也没有这个具有争议性的特性。我们建议使用thenReturn() 或thenThrow()来打桩。这两种方法足够用于测试或者测试驱动开发。
\item 实际上笔者日常开发中也不怎么用到这个特性。
\end{itemize}
\end{itemize}
\subsubsection{拦截方法返回值(常用)}
\label{sec-1-1-13}
\begin{minted}[frame=lines,fontsize=\scriptsize,linenos=false]{java}
/**
  * doReturn()、doThrow()、doAnswer()、doNothing()、doCallRealMethod()系列方法的运用
  * https://static.javadoc.io/org.mockito/mockito-core/2.8.9/org/mockito/Mockito.html#do_family_methods_stubs
  */
@Test
public void returnTest() throws Exception {
    //返回值为null的函数,可以通过这种方式进行测试

    doAnswer(invocation -> {
        System.out.println("测试无返回值的函数");
        return null;
    }).when(mockList).clear();

    doThrow(new RuntimeException("测试无返回值的函数->抛出异常"))
            .when(mockList).add(eq(1), anyString());

    doNothing().when(mockList).add(eq(2), anyString());

//        doReturn("123456").when(mockList).add(eq(3), anyString());    //不能把空返回值的函数与doReturn关联

    mockList.clear();
    mockList.add(2, "123");
    mockList.add(3, "123");
    mockList.add(4, "123");
    mockList.add(5, "123");

    //但是请记住这些add实际上什么都没有做,mock对象中仍然什么都没有
    System.out.print(mockList.get(4));
}
\end{minted}
\begin{itemize}
\item 我们不禁这样想,这些方法和when(mock.do()).thenReturn(foo)这样的方法有什么区别,或者说,这些方法有必要吗?
\item 答案是肯定的,因为在接下来介绍的新特性Spy中,该方法起到了至关重要的作用。
\item 可以说,以上方法绝对是不可代替的。
\item Mockito框架不支持mock匿名类、final类、static方法、private方法。
\item 而PowerMock框架解决了这些问题。关于PowerMock,下一篇会讲到
\end{itemize}
\subsubsection{对复杂的Mock使用Answers}
\label{sec-1-1-14}
\begin{itemize}
\item 通过Answer可以定义一个复杂的结果对象,虽然thenReturn每次都返回一个预定义的值,但是有了Answer,你可以根据stubbed方法的参数来预估响应。如果你的stubbed方法调用其中一个参数的函数,或者返回第一个参数以允许方法链的进行,那么这会很有用。后者存在一种静态方法。注意,有不同的方式来配置Answer:
\end{itemize}
\begin{minted}[frame=lines,fontsize=\scriptsize,linenos=false]{java}
import static org.mockito.AdditionalAnswers.returnsFirstArg;
@Test
public final void answerTest() {
    // with doAnswer():
    doAnswer(returnsFirstArg()).when(list).add(anyString());
    // with thenAnswer():
    when(list.add(anyString())).thenAnswer(returnsFirstArg());
    // with then() alias:
    when(list.add(anyString())).then(returnsFirstArg());
}
\end{minted}
\begin{itemize}
\item 或者如果你需要对你的结果进行回调:
\end{itemize}
\begin{minted}[frame=lines,fontsize=\scriptsize,linenos=false]{java}
@Test
public final void callbackTest() {
    ApiService service = mock(ApiService.class);
    when(service.login(any(Callback.class))).thenAnswer(i -> {
        Callback callback = i.getArgument(0);
        callback.notify("Success");
        return null;
    });
}
\end{minted}
\begin{itemize}
\item 甚至可以模拟像DAO这样的持久性服务,但是如果您的Answers过于复杂,您应该考虑创建一个虚拟类而不是mock
\end{itemize}
\begin{minted}[frame=lines,fontsize=\scriptsize,linenos=false]{java}
List<User> userMap = new ArrayList<>();
UserDao dao = mock(UserDao.class);
when(dao.save(any(User.class))).thenAnswer(i -> {
    User user = i.getArgument(0);
    userMap.add(user.getId(), user);
    return null;
});
when(dao.find(any(Integer.class))).thenAnswer(i -> {
    int id = i.getArgument(0);
    return userMap.get(id);
});
\end{minted}

\subsubsection{Spy:监控真实对象(重要)}
\label{sec-1-1-15}
\begin{minted}[frame=lines,fontsize=\scriptsize,linenos=false]{java}
/**
  * 监控真实对象
  * https://static.javadoc.io/org.mockito/mockito-core/2.8.9/org/mockito/Mockito.html#spy
  */
@Test
public void spyTest() throws Exception {
   List list = new ArrayList();
   List spyList = spy(list);

   // 当spyList调用size()方法时,return100
   when(spyList.size()).thenReturn(100);

   spyList.add("one");
   spyList.add("two");

   System.out.println("spyList第一个元素" + spyList.get(0));
   System.out.println("spyList.size = " + spyList.size());

   verify(spyList).add("one");
   verify(spyList).add("two");

   // 请注意!下面这行代码会报错! java.lang.IndexOutOfBoundsException: Index: 10, Size: 2
   // 不可能 : 因为当调用spy.get(0)时会调用真实对象的get(0)函数,此时会发生异常,因为真实List对象是空的
//         when(spyList.get(10)).thenReturn("ten");

   // 应该这么使用

   doReturn("ten").when(spyList).get(9);
   doReturn("eleven").when(spyList).get(10);

   System.out.println("spyList第10个元素" + spyList.get(9));
   System.out.println("spyList第11个元素" + spyList.get(10));

   // Mockito并不会为真实对象代理函数调用,实际上它会拷贝真实对象。因此如果你保留了真实对象并且与之交互
   // 不要期望从监控对象得到正确的结果。当你在监控对象上调用一个没有被stub的函数时并不会调用真实对象的对应函数,你不会在真实对象上看到任何效果。

   // 因此结论就是 : 当你在监控一个真实对象时,你想在stub这个真实对象的函数,那么就是在自找麻烦。或者你根本不应该验证这些函数。
}
\end{minted}
\begin{itemize}
\item Spy绝对是一个好用的功能,我们不要滥用,但是需要用到对真实对象的测试操作,spy绝对是一个很不错的选择。
\end{itemize}
\subsubsection{捕获参数(重要)}
\label{sec-1-1-16}
\begin{minted}[frame=lines,fontsize=\scriptsize,linenos=false]{java}
/**
 * 为接下来的断言捕获参数(API1.8+)
 * https:// static.javadoc.io/org.mockito/mockito-core/2.8.9/org/mockito/Mockito.html#captors
 */
@Test
public void captorTest() throws Exception {
    Student student = new Student();
    student.setName("qingmei2");

    ArgumentCaptor<Student> captor = ArgumentCaptor.forClass(Student.class);
    mockList.add(student);
    verify(mockList).add(captor.capture());

    Student value = captor.getValue();

    Assert.assertEquals(value.getName(),"qingmei2");
}

@Data
private class Student {
    private String name;
}
\end{minted}
\begin{itemize}
\item 我们将定义好的ArgumentCaptor参数捕获器放到我们需要去监控捕获的地方,如果真的执行了该方法,我们就能通过captor.getValue()中取到参数对象,如果没有执行该方法,那么取到的只能是null或者基本类型的默认值。
\item 小结
\begin{itemize}
\item 本文看起来是枯燥无味的,事实上也确实如此,但是如果想在开发中写出高覆盖率的单元测试,Mockito强大的功能一定能让你学会之后爱不释手。
\item Mockito当然也有一定的限制。例如,你不能mock静态方法和私有方法。
\end{itemize}
\end{itemize}
\subsubsection{更多的注解}
\label{sec-1-1-17}
\begin{itemize}
\item 使用注解都需要预先进行配置,怎么配置见6.2.7说明
\begin{itemize}
\item @Captor 替代ArgumentCaptor
\item @Spy 替代spy(Object)
\item @Mock 替代mock(Class)
\item @InjectMocks 创建一个实例,其余用@Mock(或@Spy)注解创建的mock将被注入到用该实例中
\end{itemize}
\end{itemize}
\subsection{容易概念混淆的几个点}
\label{sec-1-2}
\subsubsection{@Mock与@Spy的异同}
\label{sec-1-2-1}
\begin{itemize}
\item Mock对象只能调用stubbed方法,不能调用其真实的方法。而Spy对象可以监视一个真实的对象,对Spy对象进行方法调用时,会调用真实的方法。
\item 两者都可以stubbing对象的方法,让方法返回我们的期望值。
\item 两者无论是否是真实的方法调用,都可进行verify验证。
\item 对final类、匿名类、java的基本数据类型是无法进行mock或者spy的。
\item 注意mockito是不能mock static方法的。
\end{itemize}
\subsubsection{@InjectMocks与@Mock等的区别}
\label{sec-1-2-2}
\begin{itemize}
\item @Mock:创建一个mock对象。
\item @InjectMocks:创建一个实例对象,然后将@Mcok或者@Spy注解创建的mock对象注入到该实例对象中。
\item stackoverflow上对这个有一个比较形象的解释:
\end{itemize}
\begin{minted}[frame=lines,fontsize=\scriptsize,linenos=false]{java}
@RunWith(MockitoJUnitRunner.class)
public class SomeManagerTest {
    @InjectMocks
    private SomeManager someManager;
    @Mock
    private SomeDependency someDependency; // 该mock对象会被注入到someManager对象中
    // 你不用向下面这样实例化一个SomeManager对象,@InjectMocks会自动帮你实现
    // SomeManager someManager = new SomeManager();    
    // SomeManager someManager = new SomeManager(someDependency);
}
\end{minted}
\subsubsection{when(\ldots{}).thenReturn()与doReturn(\ldots{}).when(\ldots{})两种语法的异同}
\label{sec-1-2-3}
\begin{itemize}
\item 两者都是用来stubbing方法的,大部分情况下,两者可以表达同样的意思,与Java里的do/while、while/do语句类似。
\item 对void方法不能使用when/thenReturn语法。
\item 对spy对象要慎用when/thenReturn,如:
\end{itemize}
\begin{minted}[frame=lines,fontsize=\scriptsize,linenos=false]{java}
List spyList = spy(new ArrayList());
// 下面代码会抛出IndexOutOfBoundsException
when(spyList.get(0)).thenReturn("foo");
// 这里不会抛出异常
doReturn("foo").when(spyList).get(0);
System.out.println(spyList.get(0));
\end{minted}
\begin{itemize}
\item 个人觉得讨论哪种语法好是没有意义的,推荐使用doReturn/when语法,不管是mock还是spy对象都适用。
\end{itemize}

\subsection{怎样测试异步代码}
\label{sec-1-3}
\begin{itemize}
\item 异步无处不在,特别是网络请求,必须在子线程中执行。异步一般用来处理比较耗时的操作,除了网络请求外还有数据库操作、文件读写等等。一个典型的异步方法如下:
\end{itemize}
\begin{minted}[frame=lines,fontsize=\scriptsize,linenos=false]{java}
public class DataManager {
    public interface OnDataListener {
        public void onSuccess(List<String> dataList);
        public void onFail();
    }
    public void loadData(final OnDataListener listener) {
        new Thread(new Runnable() {
                @Override public void run() {
                    try {
                        Thread.sleep(1000);
                        List<String> dataList = new ArrayList<String>();
                        dataList.add("11");
                        dataList.add("22");
                        dataList.add("33");
                        if (listener != null) 
                            listener.onSuccess(dataList);
                    } catch (InterruptedException e) {
                        e.printStackTrace();
                        if (listener != null) 
                            listener.onFail();
                    }
                }
            }).start();
    }
}
\end{minted}
\begin{itemize}
\item 上面代码里开启了一个异步线程,等待1秒之后在回调函数里成功返回数据。通常情况下,我们针对loadData()方法写如下单元测试:
\end{itemize}
\begin{minted}[frame=lines,fontsize=\scriptsize,linenos=false]{java}
@Test
public void testGetData() {
    final List<String> list = new ArrayList<String>();
    DataManager dataManager = new DataManager();
    dataManager.loadData(new DataManager.OnDataListener() {
            @Override
                public void onSuccess(List<String> dataList) {
                if(dataList != null) {
                    list.addAll(dataList);
                }
            }

            @Override
                public void onFail() {
            }
        });
    Assert.assertEquals(3, list.size());
}
\end{minted}
\begin{itemize}
\item 执行这段测试代码,你会发现永远都不会通过。因为loadData()是一个异步方法,当我们在执行Assert.assertEquals()方法时,loadData()异步方法里的代码还没执行,所以list.size()返回永远是0。
\item 这只是一个最简单的例子,我们代码里肯定充斥着各种各样的异步代码,那么对于这些异步该怎么测试呢?
\item 要解决这个问题,主要有2个思路:一是等待异步操作完成,然后在进行assert断言;二是将异步操作变成同步操作。
\end{itemize}
\subsubsection{1.等待异步完成:使用CountDownLatch}
\label{sec-1-3-1}
\begin{itemize}
\item 前面的例子,等待异步完成实际上就是等待callback函数执行完毕,使用CountDownLatch可以达到这个目标,不熟悉该类的可自行搜索学习。修改原来的测试用例代码如下:
\end{itemize}
\begin{minted}[frame=lines,fontsize=\scriptsize,linenos=false]{java}
@Test
public void testGetData() {
    final List<String> list = new ArrayList<String>();
    DataManager dataManager = new DataManager();
    final CountDownLatch latch = new CountDownLatch(1);
    dataManager.loadData(new DataManager.OnDataListener() {
            @Override
                public void onSuccess(List<String> dataList) {
                if (dataList != null) 
                    list.addAll(dataList);
                // callback方法执行完毕侯,唤醒测试方法执行线程
                latch.countDown();
            }
            @Override
                public void onFail() {
            }
        });
    try {
        // 测试方法线程会在这里暂停, 直到loadData()方法执行完毕, 才会被唤醒继续执行
        latch.await();
    } catch (InterruptedException e) {
        e.printStackTrace();
    }
    Assert.assertEquals(3, list.size());
}
\end{minted}
\begin{itemize}
\item CountDownLatch适用场景:
\begin{itemize}
\item 1.方法里有callback函数调用的异步方法,如前面所介绍的这个例子。
\item 2.RxJava实现的异步,RxJava里的subscribe方法实际上与callback类似,所以同样适用。
\end{itemize}
\item CountDownLatch同样有它的局限性,就是必须能够在测试代码里调用countDown()方法,这就要求被测的异步方法必须有类似callback的调用,也就是说异步方法的调用结果必须是通过callback调用通知出去的,如果我们采用其他通知方式,例如EventBus、Broadcast将结果通知出去,CountDownLatch则不能实现这种异步方法的测试了。
\item 实际上,可以使用synchronized的wait/notify机制实现同样的功能。我们将测试代码稍微改改如下:
\end{itemize}
\begin{minted}[frame=lines,fontsize=\scriptsize,linenos=false]{java}
@Test
public void testGetData() {
    final List<String> list = new ArrayList<String>();
    DataManager dataManager = new DataManager();
    final Object lock = new Object();
    dataManager.loadData(new DataManager.OnDataListener() {
            @Override
                public void onSuccess(List<String> dataList) {
                if (dataList != null) 
                    list.addAll(dataList);
                synchronized (lock) 
                    lock.notify();
            }
            @Override
                public void onFail() {
            }
        });
    try {
        synchronized (lock) 
            lock.wait();
    } catch (InterruptedException e) {
        e.printStackTrace();
    }
    Assert.assertEquals(3, list.size());
}
\end{minted}
\begin{itemize}
\item CountDownLatch与wait/notify相比而言,语义更简单,使用起来方便很多。
\end{itemize}

\subsubsection{2. 将异步变成同步}
\label{sec-1-3-2}
\begin{itemize}
\item 下面介绍几种不同的异步实现。
\end{itemize}
\begin{enumerate}
\item 2.1 使用RxJava
\label{sec-1-3-2-1}
\begin{itemize}
\item RxJava现在已经被广泛运用于Android开发中了,特别是结合了Rotrofit框架之后,简直是异步网络请求的神器。RxJava发展到现在最新的版本是RxJava2,相比RxJava1做了很多改进,这里我们直接采用RxJava2来讲述,RxJava1与之类似。对于前面的异步请求,我们采用RxJava2来改造之后,代码如下:
\end{itemize}
\begin{minted}[frame=lines,fontsize=\scriptsize,linenos=false]{java}
public Observable<List<String>> loadData() {
    return Observable.create(new ObservableOnSubscribe<List<String>>() {
            @Override public void subscribe(ObservableEmitter<List<String>> e) throws Exception {
                Thread.sleep(1000);
                List<String> dataList = new ArrayList<String>();
                dataList.add("11");
                dataList.add("22");
                dataList.add("33");
                e.onNext(dataList);
                e.onComplete();
            }
        }).subscribeOn(Schedulers.io()).observeOn(AndroidSchedulers.mainThread());
}
\end{minted}
\begin{itemize}
\item RxJava2都是通过subscribeOn(Schedulers.io()).observeOn(AndroidSchedulers.mainThread())来实现异步的,这段代码表示所有操作都在IO线程里执行,最后的结果是在主线程实现回调的。这里要将异步变成同步的关键是改变subscribeOn()的执行线程,有2种方式可以实现:
\item 将subscribeOn()以及observeOn()的参数通过依赖注入的方式注入进来,正常运行时跑在IO线程中,测试时跑在测试方法运行所在的线程中,这样就实现了异步变同步。
\item 使用RxJava2提供的RxJavaPlugins工具类,让Schedulers.io()返回当前测试方法运行所在的线程。
\end{itemize}
\begin{minted}[frame=lines,fontsize=\scriptsize,linenos=false]{java}
@Before
public void setup() {
    RxJavaPlugins.reset();
    // 设置Schedulers.io()返回的线程
    RxJavaPlugins.setIoSchedulerHandler(new Function<Scheduler, Scheduler>() {
            @Override
                public Scheduler apply(Scheduler scheduler) throws Exception {
                // 返回当前的工作线程,这样测试方法与之都是运行在同一个线程了,从而实现异步变同步。
                return Schedulers.trampoline();
            }
        });
}
@Test
public void testGetDataAsync() {    
    final List<String> list = new ArrayList<String>();
    DataManager dataManager = new DataManager();
    dataManager.loadData().subscribe(new Consumer<List<String>>() {
            @Override
                public void accept(List<String> dataList) throws Exception {
                if(dataList != null) {
                    list.addAll(dataList);
                }
            }
        }, new Consumer<Throwable>() {
            @Override
                public void accept(Throwable throwable) throws Exception {

            }
        });
    Assert.assertEquals(3, list.size());
}
\end{minted}
\item 2.2 new Thread()方式做异步操作
\label{sec-1-3-2-2}
\begin{itemize}
\item 如果你的代码里还有直接new Thread()实现异步的方式,唯一的建议是赶紧去使用其他的异步框架吧。
\end{itemize}
\item 2.3 使用Executor
\label{sec-1-3-2-3}
\begin{itemize}
\item 如果我们使用Executor来实现异步,可以使用依赖注入的方式,在测试环境中将一个同步的Executor注入进去。实现一个同步的Executor很简单。
\end{itemize}
\begin{minted}[frame=lines,fontsize=\scriptsize,linenos=false]{java}
Executor executor = new Executor() {
    @Override
    public void execute(Runnable command) {
        command.run();
    }
};
\end{minted}
\item 2.4 AsyncTask
\label{sec-1-3-2-4}
\begin{itemize}
\item 现在已经不推荐使用AsyncTask了,如果一定要使用,建议使用AsyncTask.executeOnExecutor(Executor exec, Params\ldots{} params)方法,然后通过依赖注入的方式,在测试环境中将同步的Executor注入进去。
\item 小结
\begin{itemize}
\item 本文主要介绍了针对异步代码进行单元测试的2种方法:一是等待异步完成,二是将异步变成同步。前者需要写很多侵入性代码,通过加锁等机制来实现,并且必须符合callback机制。其他还有很多实现异步的方式,例如IntentService、HandlerThread、Loader等,综合比较下来,使用RxJava2来实现异步是一个不错的方案,它不仅功能强大,并且在单元测试中能毫无侵入性的将异步变成同步,在这里强烈推荐!
\end{itemize}
\end{itemize}
\end{enumerate}
% Emacs 27.1 (Org mode 8.2.7c)
\end{document}